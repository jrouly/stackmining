\documentclass[letterpaper,10pt]{article}

\usepackage[fleqn]{amsmath}
\usepackage{amstext}
\usepackage{dcolumn}
\usepackage{courier}
\usepackage{listings}
\usepackage{color}
\usepackage{graphicx}
\usepackage{hyperref}
\usepackage{array}
\usepackage{multirow}
\usepackage{amssymb}
\usepackage{pifont}
\usepackage{enumitem}
\usepackage[normalem]{ulem}

\definecolor{mygreen}{rgb}{0,0.6,0}
\definecolor{mygray}{rgb}{0.5,0.5,0.5}
\definecolor{mymauve}{rgb}{0.58,0,0.82}

\DeclareMathSymbol{\mlq}{\mathord}{operators}{``}
\DeclareMathSymbol{\mrq}{\mathord}{operators}{`'}

% Syntax highliting
\lstset{
    language=Python,
    basicstyle=\ttfamily\small,
    breaklines=true,
    prebreak=\raisebox{0ex}[0ex][0ex]{\ensuremath{\hookleftarrow}},
    frame=none,
    showtabs=false,
    showspaces=false,
    showstringspaces=false,
    keywordstyle=\color{Dandelion}\textbf,
    stringstyle=\color{BrickRed},
    commentstyle=\color{gray}\itshape,
    numbers=left,
    captionpos=t,
    escapeinside={\%*}{*)},
    morekeywords={minus},
}

\newcolumntype{d}[1]{D{.}{.}{#1} }

\newcommand{\fref}[1]{Figure~\ref{#1}}

\title{
  \Huge\textbf{Mining StackExchange} \\
  \LARGE Data Mining Final Project \\
  CS 484 \\
}
\author{
  Jean Michel Rouly\\
  \texttt{jrouly@gmu.edu}
  \and
  Joshua Wells\\
  \texttt{jwells@gmu.edu}
}
\date{\today}

\begin{document}

\maketitle


  \section{Problem Description}

  Give a brief but precise description or definition of the problem or
  hypothesis you set to evaluate.


  \section{Related Work}

  How does this problem and the method relate to problems/methods others
  have developed in the past.


  \section{Solution}

  How did you solve the problem? Describe the technical approach. Tell us
  what method/algorithm did you use, develop or extend and how did you
  implement it.


  \section{Experiments}

  \subsection{Data} Briefly describe the data and its size (number of
  records and number of features, type of features etc.)
  \subsection{Experimental Setup} Describe how did you setup your
  experiments, how the training/testing data was prepared, what performance
  metrics are you considering, what baseline methods for comparison are you
  using.
  \subsection{Experimental Results} Describe your experimental results.
  Structure your experiments around particular aspects of your method. For
  example, you could structure the experiment as follows: (1) a table
  showing results of your method using different types of features; (2)
  table comparing the performance of your method to the baseline; (3) a
  graph plotting the size of the training dataset vs. the time it takes to
  train the model; (4) Investigation of the learned model (what are the
  important features, etc.).


  \section{Conclusion}


  \section{Contributions}

  \subsection{Joshua's Contributions}

  \subsection{Michel's Contributions}



\end{document}
