\documentclass[letterpaper,10pt]{article}

\usepackage{amsmath}
\usepackage{amstext}
\usepackage{dcolumn}
\usepackage{courier}
\usepackage{listings}
\usepackage{color}
\usepackage{graphicx}
\usepackage{hyperref}
\usepackage{array}
\usepackage{multirow}
\usepackage{amssymb}
\usepackage{pifont}
\usepackage{enumitem}
\usepackage[normalem]{ulem}

\definecolor{mygreen}{rgb}{0,0.6,0}
\definecolor{mygray}{rgb}{0.5,0.5,0.5}
\definecolor{mymauve}{rgb}{0.58,0,0.82}

\DeclareMathSymbol{\mlq}{\mathord}{operators}{``}
\DeclareMathSymbol{\mrq}{\mathord}{operators}{`'}

% Syntax highlighting
\lstset{
    language=Python,
    basicstyle=\ttfamily\small,
    breaklines=true,
    prebreak=\raisebox{0ex}[0ex][0ex]{\ensuremath{\hookleftarrow}},
    frame=none,
    showtabs=false,
    showspaces=false,
    showstringspaces=false,
    keywordstyle=\color{Dandelion}\textbf,
    stringstyle=\color{BrickRed},
    commentstyle=\color{gray}\itshape,
    numbers=left,
    captionpos=t,
    escapeinside={\%*}{*)},
    morekeywords={minus},
}

\newcolumntype{d}[1]{D{.}{.}{#1} }

\newcommand{\fref}[1]{Figure~\ref{#1}}

\title{
  \Huge\textbf{Mining StackExchange} \\
  \LARGE Data Mining Final Project \\
  CS 484 \\
}
\author{
  Jean Michel Rouly\\
  \texttt{jrouly@gmu.edu}
  \and
  Joshua Wells\\
  \texttt{jwells@gmu.edu}
}
\date{\today}

\begin{document}

\maketitle


% 1. Problem Description (Done) {{{
\section{Problem Description}

Textual data is everywhere on the Internet today. This written language
data comes in a wide variety of differing sizes, structures, reading
levels, languages, and character sets. Identifying patterns and
sub-structure in a large, unknown data space like this is a formidable
problem. And yet, because of textual data's ubiquity and value, it is one
of the most important problems to solve efficiently and effectively. Data
mining tools based on machine learning techniques provide a possible
solution to this problem.

Because of the unknown nature of large, Internet sourced textual
datasets, the most readily applicable algorithms for data investigation
are ``unsupervised'' clustering algorithms. These algorithms are
considered unsupervised because, generally, no initial knowledge about
the dataset is required. Thus they serve as excellent tools to perform
preliminary investigation on an unknown dataset. However, given knowledge
about the dataset (\textit{e.g.} a ground-truth set of labels over the
data) other algorithms become relevant. There are a variety of powerful
classification algorithms which can be applied to large textual data sets
when the labels of a subset of training data are known beforehand.

The goal of this project is to apply a variety of data mining techniques,
specifically clustering and classifying algorithms, to a large dataset of
textual, natural-language data. The dataset used is the archived collection
of posts from StackExchange.\footnote{\url{http://stackexchange.com}}
Several different algorithms will be applied to determine which
StackExchange site an unseen Post belongs to, and their performance
compared on this prototypical Internet sourced textual dataset. A secondary
goal of this investigation is to showcase some of the features of the
Python machine learning library
\texttt{scikit-learn}.\footnote{\url{http://scikit-learn.org}} After the
experiments are completed, evaluation metrics will provide insight into the
algorithmic performance.
% }}}


% 2. Related Work (Done) {{{
\section{Related Work}

This work is entirely building upon the efforts of others. The clustering
and classification algorithms used have been introduced over the years
since as early as 1963. Their implementations are all provided by the
\texttt{scikit-learn}\cite{scikit-learn} library.

% Unsupervised Methods {{{
\subsection{Unsupervised Methods}

\paragraph{Ward's Hierarchical Clustering} Ward's method\cite{ward1963} is
an early example of a hierarchical clustering process. The algorithm
operates by minimizing the per-cluster variance (sum of squared difference)
at each level. It can be efficiently scaled at the cost of more extensive
preprocessing.

\paragraph{K-Means} K-Means\cite{macqueen1967} has been one of the most
commonly used clustering algorithms since its introduction in 1967. It
operates by iteratively updating cluster prototypes as the computed means
of cluster members. K-Means clustering seeks to minimize the per-cluster
sum of squared error.

\paragraph{Mean-Shift} The Mean-Shift\cite{fukunaga1975} algorithm,
introduced in the mid 1970s, operates in a similar fashion to K-Means. One
of the primary differences is that Mean-Shift uses a kernel function to
determine the weights of nearby points for mean calculation.

\paragraph{DBSCAN} Density-based spatial clustering of applications with
noise\cite{ester1996} is a density-based clustering algorithm that is
insensitive to noise. This means that densely populated areas of the
dataset (\textit{i.e.} areas where a number of data points lie within a set
similarity threshold) are clustered together.

\paragraph{Spectral Clustering} Spectral Clustering\cite{shi1997},
introduced in 1997, begins by embedding the affinity matrix of the dataset
in a lower dimensionality space, where a secondary clustering process can
occur. Often (as in \texttt{scikit-learn}) K-Means is used in this final
step.

\paragraph{Affinity Propagation} Introduced in 2007, Affinity
Propagation\cite{frey2007} (AP) is one of the more unique clustering
mechanisms used in this study. AP runs as a message passing framework over
a similarity matrix of the dataset. Unlike K-Means, AP does not require K
as an initial input.  Interestingly, AP can be implemented to run
efficiently on a parallel computing ``cloud'' architecture.\cite{rose2013}
%}}}

% Supervised Methods {{{
\subsection{Supervised Methods}

\paragraph{Decision Trees} Decision Trees are an umbrella category of
various supervised learning methods. The algorithms operate by analyzing
the input dataset attributes for common characteristics and constructing a
tree of paths to predict the label of a novel input.

\paragraph{Random Forest} Random Forests\cite{breiman2001} are an ensemble
method combining multiple decision tree classifiers built from different
random samples of data. In the \texttt{scikit-learn} implementation,
predictions from the ensemble are averaged instead of tallied as votes.


\paragraph{Naive Bayes} Relying on Bayes' Law, Naive Bayesian
classification is a statistical learning method based on the assumption
that features occur independently. As a statistical method, Naive Bayesian
classification performs comparatively well on unknown data sets.

\paragraph{K Nearest Neighbors} K-Nearest Neighbors operates by analyzing
the neighborhoods of data nodes, building clusters based on the number of
neighbors surrounding. Because K neighbors are selected, the algorithm is
highly sensitive to user input tuning.

\paragraph{Support Vector Machines} Support Vector Machines
(SVM)\cite{vapnik1995} are a supervised learning tool used to find the
optimal decision boundary (hyperplane), as defined by the largest margin
hyperplane with the lowest error. This boundary is found by operations
through a kernel function transforming the input space into a higher
dimensional space.
% }}}

% }}}


% 3. Solution (Empty) {{{
\section{Solution}

{\color{red}
How did you solve the problem? Describe the technical approach. Tell us
what method/algorithm did you use, develop or extend and how did you
implement it.
}

% }}}


% 4. Experiments (Done) {{{
\section{Experiments}


% 4.1. Data (Done) {{{
\subsection{Data}

The dataset was 90 GB of archived textual natural-language data from
StackExchange,\footnote{\url{http://stackexchange.com}} a forum-like
website with 111 subforums (sites). Each site has a specific topic
(\textit{e.g.} programming, chess, Italian, \textit{etc.}). Each site
follows a question-answer format, where the dataset is composed entirely of
question posts. Data classes correspond to site category, data points
correspond to individual posts on the sites.

To make the data more manageable, the largest class
(\texttt{stackoverflow}) and the two smallest classes (\texttt{italian},
\texttt{ebooks}) were eliminated to help center the size of each class.
The resulting information was 108 classes (sites) and 5.2 GB of textual
information. After these initial steps, but before random sampling, the raw
data included 4,421,754 entries, and many millions of features (the entire
feature space was never measured).
% }}}


% 4.2. Experimental Setup (Done) {{{
\subsection{Experimental Setup}

To preprocess the data, we developed a configurable Python framework to
stream in data from an outside location (\texttt{disk}: on a local or
networked block storage device; or \texttt{s3}: from a remote Amazon S3
bucket) and parse a random sample (without replacement) into memory.
Streaming from Amazon S3 used the
\texttt{Boto}\footnote{\url{https://github.com/boto/boto}} library.

Each site's \texttt{Posts.xml} was parsed iteratively using
\texttt{lxml}\footnote{\url{http://lxml.de}} to interpret the XML
structure. Rows (posts) were randomly sampled at a user-specified count
from this file. Selected posts were stripped of any HTML elements using
\texttt{lxml}, non-ASCII characters, and empty strings.

Once the entire dataset is read into memory, \texttt{scikit-learn} is used
to perform TF-IDF vectorization. Different values of \texttt{max\_df} and
\texttt{min\_df} were tried out to minimize the feature set while still
retaining enough useful, identifying attributes. We settled on three
different subsets of the data: ``small,'' ``medium,'' and ``large.'' See
Table \ref{tbl:tfidf} for details.


\begin{table}[ht]
\center
\begin{tabular}{cccccc}
Label & Sample Size & Data Points & Dimensionality & \texttt{max\_df} &
\texttt{min\_df} \\
\hline
Small & 100 & 10,800 & 5,904 & 90\% & 10 \\
Medium & 500 & 53,908 & 11,319 & 80\% & 20 \\
Large & 1000 & 106,908 & 11,416 & 75\% & 40 \\
\hline
\end{tabular}
\caption{Sampled Dataset Detail}
\label{tbl:tfidf}
\end{table}


\noindent
Note that random sampling and re-vectorization occurs every time the
framework executes so as to prevent repetition over the same sample.
Multiple analytics can be chained together over a single execution, though.

We ran these datasets through several algorithms provided by
\texttt{scikit-learn}. Initially, we used K-means, DB-scan, Affinity
Propagation, Spectral Clustering, and Ward's Method (hierarchical
clustering). In order to operate on such a large dataset we provisioned an
Amazon EC2 cluster (\texttt{r3.4xlarge}, see Table \ref{tbl:ec2} for
details). The EC2 cluster ran an initial provisioning script to setup the
environment, stream in data from S3, and compile dependencies. The EC2
cluster was used to run all experiments and metrics.


\begin{table}[ht]
\center
\begin{tabular}{cc}
\multicolumn{2}{c}{\texttt{r3.4xlarge}} \\
\hline
RAM & 122 GB \\
Disk & 1x320 SSD \\
Cores & 16 \\
ECU & 52 \\
\hline
\end{tabular}
\caption{Amazon EC2 Detail}
\label{tbl:ec2}
\end{table}


For each experiment, runtime of the algorithm and performance metrics were
recorded. For clustering algorithms, this meant V-measure, homogeneity, and
completeness.  V-measure was selected as the representative statistic of
clustering performance because it combined the two subsidiary metrics (see
Eq.  \ref{eq:vmeasure}). All three metrics require ground-truth knowledge,
and V-measure can be (very) roughly approximated as accuracy in order to
compare clustering performance with classification performance, which is
implemented next.


\begin{align}
h = 1 - \frac{H(C|K)}{H(C)}\quad;\quad
c = 1 - \frac{H(K|C)}{H(K)}\quad;\quad
v = 2 \cdot \frac{h \cdot c}{h + c}
\label{eq:vmeasure}
\end{align}


Once results for clustering algorithms were obtained, we added
classification tools to the machine learning framework developed so far.
With \texttt{scikit-learn} we provided Decision Trees, Random Forests,
Naive Bayesian classification, Support Vector Machines, and K Nearest
Neighbor classification. Each run, input data was split into 60\% training
data and 40\% testing data.


Many of the machine learning functions required input parameter tuning from
the user. K-Means, Spectral clustering, and Ward's Method were all seeded
with the true number of classes as the target number of clusters. Damping
for Affinity Propagation was set to 0.5 to accelerate convergence. DBSCAN
was initialized with $\epsilon=0.01$ and \texttt{min\_samples}$=1$. We
believe that this resulted in garbage output for DBSCAN, but no other
initialization values allowed the algorithm to cluster. Decision Trees
consumed the most resources but performed the best when initialized with
default settings; attempts to trim the depth of the tree only decreased
performance. A range of different $K$ values were sampled for KNN, but the
default value of 5 provided the best results, despite massively slow
runtime. Finally, we left SVM with a linear kernel for the sake of
simplicity.
% }}}


% 4.3. Experimental Results (Done) {{{
\subsection{Experimental Results}

% Clustering {{{
\subsubsection{Clustering}

The most readily scalable algorithm was K-Means. DBSCAN was slower than
K-Means on every size dataset, but was still able to handle even the large
data set. Affinity Propagation, Spectral Clustering, and Ward's Method all
took enormous amounts of time for even the small dataset and ended up not
being able to handle the larger datasets at all within a reasonable amount
of time (or else faced data structure overflow).

It is important to also consider how well the clustering algorithms assign
posts to clusters. Considering V-measure, the worst performer was Affinity
Propagation, achieving only 0.15. Part of the reason why Affinity
Propagation might be bad for this type of data is that there is no inherent
graph or network structure to the data Considering the poor
resulting predictions compared to the other methods and the problem with
scaling, it seems like this is not the ideal clustering technique for
trivially processed textual data. Perhaps with a better fit similarity
metric, or with better preprocessing techniques (specifically stemming and
lexical analysis) Affinity Propagation might perform better.

The next worst performer is K-Means. K-Means' V-measure values dropped
significantly as the dataset got larger, from slightly over 0.36
for the smallest data set to under 0.3 for the largest set. Part of the
issue might be that the larger sets are more likely to be caught in local
minima. Additionally, the larger datasets contained significantly more
features so the curse of dimensionality could be another issue. Thus, it
might be best to reduce the features through a dimensionality reduction
technique like PCA or similar. Regardless, K-Means is the best performer
in terms of run time, and scales the most rapidly to large data sets.

Both Spectral clustering and Ward's Method obtained similar V-measures
(almost 0.4 for the small dataset) and took significantly longer to produce
results than K-Means or DBSCAN. Both of these methods might be valuable on
far smaller data sets, but neither scale well to big data.

Finally, DBSCAN produced, by far, the most high quality results. It
achieved a V-measure of about 0.6 with little variance on the different
sizes of datasets. While it did take far longer than K-means to execute, it
was still able to handle the largest dataset. Part of the reason it might
perform so well is that it can detect differently sized clusters and
doesn't make assumptions about cluster shape. On the other hand, the high
V-measure values may also indicate overfitting to the data (which is highly
possible given that it was initialized with $\epsilon=0.01$). See Table
\ref{tbl:clustering} for a complete record of results.


\begin{table}[ht]
\center
\begin{tabular}{cccccc}
Dataset & Algorithm & Runtime & \textbf{V} & \textbf{H} & \textbf{C} \\
\hline
\multicolumn{1}{c}{\multirow{6}{*}{Small}}
      & DBSCAN & 57.90 s & 0.60 & 1.00 & 0.43 \\
      & Spectral & 1334.91 s & 0.39 & 0.39 & 0.39 \\
      & Ward's & 1613.21 s & 0.39 & 0.34 & 0.46 \\
      & K-Means & 10.68 s & 0.38 & 0.36 & 0.40 \\
      & AP & 558.10 s & 0.16 & 0.15 & 0.17 \\
      & Mean-Shift & -- & -- & -- & -- \\
\hline
\multicolumn{1}{c}{\multirow{2}{*}{Medium}}
      & DBSCAN & 1088.72 s & 0.601 & 1.00 & 0.43 \\
      & K-Means & 30.92 s & 0.35 & 0.35 & 0.32 \\
\hline
\multicolumn{1}{c}{\multirow{2}{*}{Large}}
      & DBSCAN & 2835.58 s & 0.576 & 1.00 & 0.40 \\
      & K-Means & 40.84 s & 0.29 & 0.28 & 0.32 \\
\hline
\end{tabular}
\caption{Clustering Results}
\label{tbl:clustering}
\end{table}
% }}}


% Classification {{{
\subsubsection{Classification}

The worst performing classifier by far was K-Nearest Neighbors. Since it
uses the training data for predicting, it could not scale up to the larger
sets.  Additionally, the predicting results were horrible at less than 10\%
accuracy, even when a number of different K values were tried. Overall,
this is probably a bad classifier for the problem due to the fact the
feature space is so large, and that there might be quite a few neighbors in
different classes. It is possible that the results would improve by
weighting the nearest points instead of using uniform weights.

Decision Trees proved interesting to work with. It seems that
\texttt{scikit-learn} does not provide for pruning in the basic Decision
Tree implementation, but still offer a few options to play around with,
namely the \texttt{max\_depth} of the tree. We tried a number of values
from $[5, 20]$ but the results were rather disappointing, at less than 20\%
accuracy. Using the default depth value causes the tree to split until it
either reaches a pure node or hits a certain minimum number of values. This
proved highly successful, with almost 30\% accuracy in the smallest set and
over 40\% in the medium dataset. The reason why the results probably
improved is that it could pick out more specific cases since the larger
datasets have a larger dimensionality. Since many of the StackExchange
sites have specialized topics and terminology, this Decision Tree structure
must do a good job at identifying these features. Consequently, though, it
probably suffers from overfitting to the data. Also, the runtime ended up
taking significantly longer than any of the other algorithms.

A similar algorithm to Decision Trees is the Random Forest ensemble method.
Since the default parameters worked well for vanilla Decision Trees, we
decided to let Random Forest classification initialize with default values
as well. The results were significantly improved over Decision Trees, with
accuracy of 46\% on the small dataset and up to 51\% on the largest. Part
of the reason why the results were improved is that the Random Forest helps
to combat overfitting as an ensemble method. A few individual trees might
overfit for some cases (increasing bias), but the majority would more
likely not (decreasing variance through averaging results). Perhaps
surprising is that the algorithm executed more quickly than the vanilla
Decision Tree. However, it was still the second longest to construct out
of all the classifiers.

Support Vector Machines correctly predicted classes, even with default
initialization settings and a linear kernel, the most effectively of all
attempted methods. It was able to achieve accuracy close to 70\% for all
three data sets. Additionally, its execution time was among the fastest.
Its high quality performance indicates that the data must be relatively
easily linearly separable. Some of the other kernel functions might produce
even better results, but we did not investigate them. See Table
\ref{tbl:classification} for a complete record of results.


\begin{table}[ht]
\center
\begin{tabular}{cccccc}
Dataset & Algorithm & Runtime & Accuracy & Precision & Recall \\
\hline
\multicolumn{1}{c}{\multirow{5}{*}{Small}}
      & SVM & 1.42 s & 0.69 & 0.70 & 0.69 \\
      & Random Forests & 12.96 s & 0.46 & 0.50 & 0.46 \\
      & Naive Bayesian & 0.75 s & 0.51 & 0.57 & 0.51 \\
      & Decision Trees & 124.91 s & 0.29 & 0.31 & 0.29 \\
      & KNN & 1.50 s & 0.04 & 0.48 & 0.04 \\
\hline
\multicolumn{1}{c}{\multirow{4}{*}{Medium}}
      & SVM & 17.13 s & 0.72 & 0.72 & 0.72 \\
      & Random Forests & 155.99 s  & 0.50 & 0.52 & 0.50 \\
      & Naive Bayesian & 7.28 s    & 0.49 & 0.51 & 0.49 \\
      & Decision Trees & 1924.73 s & 0.41 & 0.42 & 0.41 \\
\hline
\multicolumn{1}{c}{\multirow{3}{*}{Large}}
      & SVM & 28.61 s & 0.71 & 0.71 & 0.71 \\
      & Random Forests & 329.01 s & 0.51 & 0.52 & 0.51 \\
      & Naive Bayesian & 14.93 s  & 0.44 & 0.46 & 0.43 \\
\hline
\end{tabular}
\caption{Classification Results}
\label{tbl:classification}
\end{table}
% }}}


% }}}


% }}}


% 5. Conclusions {{{
\section{Conclusion}

% }}}


% 6. Contributions {{{
\section{Contributions}

\subsection{Joshua's Contributions}
\subsection{Joshua's Contributions}


\subsection{Michel's Contributions}
\subsection{Michel's Contributions}

% }}}


\clearpage
\bibliographystyle{acm}
\bibliography{rouly_wells_writeup}


\end{document}
